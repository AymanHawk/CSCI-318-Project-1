\documentclass[pdf]{beamer}
\mode<presentation>{\usetheme{CambridgeUS}}
\usecolortheme{beaver}
\useinnertheme{rounded}
\useoutertheme{tree}
\usepackage{listings}
\usepackage{pythontex}
\lstset{
 language=Python, % Set the language to Python
 basicstyle=\ttfamily, % Use a monospace font
 keywordstyle=\color{orange},
 stringstyle=\color{blue},
 commentstyle=\color{red},
 breaklines=true, % Allow line breaks
 frame=single, % Add a frame around the code 
}
\title[LaTeX Presentation]{Python in Latex?} %[x] means x is the short title
\author{Ayman Haque, Mahdi Ivan, Jaden Walker, Elizabeth Stehnach}
\date{2/20/24}
\begin{document}
\begin{frame}
 \titlepage
\end{frame}
\begin{frame}{Table of Contents}
 \tableofcontents
\end{frame}
\section{What is Python?}
\begin{frame}{What is Python?}
\begin{itemize}
 \item{Python is a high-level, object-oriented programming language}
 \item {Python features an easy-to-learn syntax, making it accessible to beginners and experienced programmers. }
 \item{It offers seamless integration with other languages, allowing developers to use existing codebases written in other languages such as Java or C++}
 \item{Python is known for providing a wide range of built-in functions and modules for various tasks such as text processing, web development, and data analysis.}
\end{itemize}
\end{frame}
\subsection{Basic Example of a Python Code}

\begin{lstlisting}[caption={Python Example}]
list = ["Welcome,","this","is", "a", "list"]
string = "This creates a new string"
for i in range(100):
 print("Hello, Class of CSCI318!")
def example(string):
 print(string)

example("Here is an example of a function")
\end{lstlisting}
\begin{frame}{Table of Contents}
 \tableofcontents
\end{frame}
\section{What is PythonTeX?}
\begin{frame}{What is PythonTeX?}
\begin{itemize}
 \item{PythonTeX is a LaTeX package that enables the execution of Python code within LaTeX documents. }
 \item{It supports a wide range of programming languages, including Python, JavaScript, Octave, etc.}
 \item{PythonTeX is fully open source and free to use for the public}
 \item{For detailed documentation and further information, users can refer to the official PythonTeX repository on GitHub: "\href{https://github.com/gpoore/pythontex}{https://github.com/gpoore/pythontex}". }
\end{itemize}
\end{frame}
\subsection{Why should you use PythonTeX?}
\begin{frame}{Why should you use PythonTeX?}
\begin{itemize}
 \item{Python can make even the hardest math problems easier. }
 \item{Using Python you can even repeat latex code.}
 \item{It can even automate document generation.}
 \item {Python easily solves issues Latex would not be able to solve.}
\end{itemize}
\end{frame}
\subsection{Example}
\begin{frame}{Example}
You are given a problem that says to find the number of prime numbers from a certain range, Let's say 1 to X. How would you tackle this problem?
\begin{itemize}
 \item{First, you would try to store the value of X.}
 \item{Need a way to check each number to determine if it is a prime}
 \item{Finally figure out what numbers are prime.}
\end{itemize}
Is this possible with latex?
\end{frame}
\subsection{Solution}
\begin{frame}[fragile]{Solution}
\begin{lstlisting}[caption={Finding the Prime Numbers}, basicstyle=\tiny]
def is_prime(k):
    """Check if a number is prime"""
    if k <= 1:
        return False
    elif k <= 3:
        return True
    elif k % 2 == 0 or k % 3 == 0:
        return False
    i = 5
    while i * i <= k:
        if k % i == 0 or k % (i + 2) == 0:
            return False
        i += 6
    return True

def getPrime(x):
    """Print all prime numbers up to x"""
    print("Prime numbers from 1 to", x, "are:")
    for i in range(1, x + 1):
        if is_prime(i):
            print(i, end=" ")
\end{lstlisting}
PythonTex makes our life so much simpler allowing us to do massive calculations within the Latex software
\end{frame}
\begin{frame}{Table of Contents}
 \tableofcontents
\end{frame}
\section{How to Implement Python into Latex}
\begin{frame}[fragile]{How to Implement Python into Latex}
When you are implementing PythonTeX you need to give the package name and a special parameter that allows you to embed Python code directly within a LaTeX document.
\begin{lstlisting}
\documentclass{article}
\usepackage{pythontex}
\begin{document}
\begin{pycode}
x = 100
for i in range (x):
 print("Hello, World!")
\end{pycode}
\end{document}
\end{lstlisting}
\end{frame}
\subsection{Explanation of how PythonTeX is Used}
\begin{frame}[fragile]{Explanation of how PythonTeX is Used}
This calls the package PythonTeX allowing latex to utilize python 
\begin{lstlisting}
\usepackage{pythontex}
\end{lstlisting}
This part of the code allows latex to know where to find your python code
\begin{lstlisting}
\begin{pycode}
# Here would be where your Python code will be implemented
\end{pycode}
\end{lstlisting}
Allows us to use the python within Latex
\begin{lstlisting}
\py{ % Here you would call functions/variables}
\end{lstlisting}
\end{frame}
\begin{frame}{Table of Contents}
 \tableofcontents
\end{frame}
\section{How to solve the Example}
\begin{frame}[fragile]{How to solve the Example}
\tiny
\begin{lstlisting}
\begin{pycode}
def is_prime(k):
    """Check if a number is prime"""
    if k <= 1:
        return False
    elif k <= 3:
        return True
    elif k % 2 == 0 or k % 3 == 0:
        return False
    i = 5
    while i * i <= k:
        if k % i == 0 or k % (i + 2) == 0:
            return False
        i += 6
    return True

def getPrime(x):
    """Print all prime numbers up to x"""
    print("Prime numbers from 1 to", x, "are:")
    for i in range(1, x + 1):
        if is_prime(i):
            print(i, end=" ")
\end{pycode}
\py{getPrime(1000)}
\end{lstlisting} 
\end{frame}
\subsection{Did it work?}
\begin{frame}[fragile]{Did it work?}
\begin{pycode}
def is_prime(k):
    """Check if a number is prime"""
    if k <= 1:
        return False
    elif k <= 3:
        return True
    elif k % 2 == 0 or k % 3 == 0:
        return False
    i = 5
    while i * i <= k:
        if k % i == 0 or k % (i + 2) == 0:
            return False
        i += 6
    return True

def getPrime(x):
    """Print all prime numbers up to x"""
    print("Prime numbers from 1 to", x, "are:")
    for i in range(1, x + 1):
        if is_prime(i):
            print(i, end=" ")
\end{pycode}
\py{getPrime(1000)}

\end{frame}

\end{document} 